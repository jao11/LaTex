\documentclass [a4paper, 12pt]{article}

\usepackage[T1]{fontenc}
\usepackage[utf8]{inputenc}
\usepackage[letterpaper,top=2cm,bottom=2cm,left=3cm,right=3cm,marginparwidth=1.75cm]{geometry}
\usepackage{hyphenat}
\hyphenation{mate-mática recu-perar}
\usepackage[brazil]{babel}
\usepackage{amsmath}
\usepackage{indentfirst}
\usepackage{calligra}
\usepackage{hyperref}

\hypersetup{
    colorlinks=true,
    linkcolor=black,
    filecolor=magenta,      
    urlcolor=black,
    pdftitle={Overleaf Example},
    pdfpagemode=FullScreen,
    }

\title{Resenha "O papel "articulador" do conceito de incerteza de medição na rede metrológica internacional"}
\author{João Pedro Silva dos Santos}
\date{29 de setembro de 2021}

\begin{document}

\maketitle

{\ttfamily{Matéria}}: Filosofia da física

{\ttfamily{Tarefa}}: Escrever uma resenha sobre o artigo: O papel "articulador" do conceito de incerteza de medição na rede metrológica internacional. 

\section*{- Título do artigo:}

O papel “articulador” do conceito de incerteza de medição na rede metrológica internacional

\section*{-  Bibliografia:}

DAVID, Mariano. O papel “articulador” do conceito de incerteza de medição na rede metrológica internacional. -, [s. l.], [2021?] ano provável. Disponível em: \url{https://ava.pr1.uerj.br/pluginfile.php/283119/mod_resource/content/1/David%2C%20Mariano_O%20papel%20%E2%80%9Carticulador%E2%80%9D%20do%20conceito%20de%20incerteza%20de%20medic%CC%A7a%CC%83o%20na%20rede%20metrolo%CC%81gica%20internacional_2021.pdf}. Acesso em: 29 set. 2021

\section*{- Objetivo do autor:}

Este artigo teve como meta mostrar um aspecto da medição que, embora central, nem sempre é ressaltado, a incerteza. E que ela não representa apenas um aspecto essencial das medições, mas que, na rede metrológica internacional, a incerteza de medição desempenha um papel “articulador” entre os resultados de medição.

\section*{- Questões do autor:}

\begin{itemize}
    \item Como se deu e porque ocorreu a implementação da metrologia?
    \item Qual a diferença entre incerteza e erro de medição?
    \item Como foi desenvolvido o sistema para estimar as incertezas experimentais?
    \item Por que desenvolver uma base lógica para as quantificações?
    \item O que é e como funciona o coerentismo?
    \item Qual a função dos modelos teóricos e por que eles surgem? 
\end{itemize}

\section*{- Referencial teórico:}

A discussão foi baseada principalmente em documentos oficiais das instituições metrológicas internacionais e em trabalhos de filosofia das medições publicados nas duas últimas décadas.

\section*{- Quais os principais assuntos abordados ao longo do texto:}

Na primeira seção, foi abordada de forma introdutória o desenvolvimento das medições e a institucionalização da metrologia, além de esclarecermos alguns conceitos básicos. Na segunda seção, realizou-se um breve relato da adoção do conceito de incerteza de medição e das recomendações do Guia para expressão da incerteza de medição (GUM). Na terceira seção, apresentou-se um painel das filosofias da medição para, em seguida, discutir conceitos e tendências centrais nos trabalhos de filosofia das medições produzida nas primeiras décadas do século XXI. Com base nestes trabalhos, argumentamos que o conceito de incerteza de medição desempenha um papel “articulador” na rede metrológica internacional. A quarta seção tratou da coerência do sistema metrológico. E na quinta seção foi visto o papel dos modelos teóricos bem como alguns exemplos. 

\section*{- Quais as conclusões do autor:}

Para a conclusão do autor deixo suas próprias palavras falarem por ele pois como ele mesmo diz nenhuma analogia poderia representar fielmente a estrutura de rede metrológica.

“O que “amarra” e dá coerência à rede metrológica é sua própria estrutura: seus nós, seus pontos de conexão. Se estes nós não fossem, de alguma forma, flexíveis, dificilmente a estrutura poderia perseverar. Vamos supor que, numa rede constituída por medições absolutamente exatas, o aprimoramento de uma metodologia de medição fornecesse um valor diferente daquele consagrado para uma constante fundamental. Toda a estrutura poderia ser abalada, pois a alteração de um valor implicaria na alteração de muitos outros. Ao invés de conferir estabilidade, a rigidez tornaria a rede frágil. Para usar uma analogia com a engenharia, as incertezas de medição funcionam como juntas de dilatação entre os braços da rede: elas permitem a flexibilidade necessária para manter a estrutura coesa. Porém, nenhuma analogia deste tipo poderia representar fielmente a estrutura da rede que é bem mais complexa do que a imagem que dela possamos fazer.”

\section*{- Minhas opiniões:}

Este é um texto que nos traz conceitos e uma parte da história por traz do conceito de metrologia. É uma fonte valiosa que nos leva a entender a importância da teoria estatística e também o esforço envolvido na teoria, para que assim pudéssemos ter mais certeza dos resultados encontrados em laboratórios além de uma uniformidade nas respostas.

\end{document}
