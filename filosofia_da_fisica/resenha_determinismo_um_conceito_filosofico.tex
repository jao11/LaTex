\documentclass [a4paper, 12pt]{article}

\usepackage[T1]{fontenc}
\usepackage[utf8]{inputenc}
\usepackage[letterpaper,top=2cm,bottom=2cm,left=3cm,right=3cm,marginparwidth=1.75cm]{geometry}
\usepackage{hyphenat}
\hyphenation{mate-mática recu-perar}
\usepackage[brazil]{babel}
\usepackage{amsmath}
\usepackage{indentfirst}
\usepackage{calligra}
\usepackage{hyperref}

\hypersetup{
    colorlinks=true,
    linkcolor=black,
    filecolor=magenta,      
    urlcolor=black,
    pdftitle={Overleaf Example},
    pdfpagemode=FullScreen,
    }

\title{Resenha - "Determinismo um conceito filosófico"}
\author{João Pedro Silva dos Santos}
\date{29 de outubro de 2021}

\begin{document}

\maketitle

{\ttfamily{Matéria}}: Filosofia da física

{\ttfamily{Tarefa}}: Escrever uma resenha sobre a aula: Determinismo um conceito filosófico de Carlos Fils Puig

\section*{- Título da aula:}

Determinismo um conceito filosófico

\section*{-  Bibliografia:}

PUIG, Carlos. Determinismo um conceito filosófico. 2021. Nota de Aula (Graduação, Física) - Professor, [S. l.], 2021. pdf.

\section*{- Objetivo do autor:}

Ao comentar sobre o que são determinismo e eventos estocásticos o autor tenta nos guiar a pensar sobre o fim das ciências mais em especifico a física.

\section*{- Questões do autor:}


\begin{itemize}
    \item O que é o determinismo?
    \item Por que apesar do determinismo das ciências surgem tantas teorias com respostas estocásticas?
    \item Como os padrões absolutos estão caindo?
    \item Podemos confiar no determinismo?
\end{itemize}

\section*{- Referencial teórico:}

O autor cita autores como Laplace, Maxwell e Boltzmann, além de comentar sobre as teorias de evolução, dos fluídos e meteorologia, da lógica formal, da mecânica quântica, das teorias de relatividade geral e restrita. Fala também do teorema da incompletude de Gödel, das teorias sociais na economia e sociologia e por fim da psicanálise e da psicologia.

\section*{- Quais os principais assuntos abordados ao longo do texto:}

Está aula começa definindo certos padrões de pensamentos para o homem moderno comum. E segue ao tentar explicar o que é o determinismo e sobre como as ciências se apoiaram nele ao longo do tempo, mesmo que somente ele não fosse o suficiente para explicar tudo. Passamos então a discutir os dados estocásticos e algumas teorias que tem dados estocásticos.

\section*{- Quais as conclusões do autor:}

Temos um começo de conclusão quando o autor argumenta que por causa da aleatoriedade dos dados de novas teorias alguns padrões deixam de ser absolutos, apesar de segundo ele mesmo não haver motivos para desconsiderar o determinismo. Então nos propõe dois experimentos mentais, o sistema fechado e o pequeno ratinho na floresta, e fecha a aula nos mostrando que toda essa situação entre o determinismo e os dados estocásticos formam um belo paradoxo. 

\section*{- Minhas opiniões:}

Posso dizer que entendo quando ele diz que enfrentamos um paradoxo entre o determinismo o os dados da natureza, que fornecem dados aleatórios, mas não concordo quando ele diz que a física ou das ciências da natureza estão próximos pois como ele mesmo ressalta, “embora em muitas instâncias não se consiga traçar a corrente de causas e efeitos de um fenômeno, isso não significa que não está lá. ”. Então posso dizer que não estamos no fim, mas sim no início das descobertas. 

\end{document}
