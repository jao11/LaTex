\documentclass [a4paper, 12pt]{article}

\usepackage[T1]{fontenc}
\usepackage[utf8]{inputenc}
\usepackage[letterpaper,top=2cm,bottom=2cm,left=3cm,right=3cm,marginparwidth=1.75cm]{geometry}
\usepackage{hyphenat}
\hyphenation{mate-mática recu-perar}
\usepackage[brazil]{babel}
\usepackage{amsmath}
\usepackage{indentfirst}
\usepackage{calligra}
\usepackage{hyperref}

\hypersetup{
    colorlinks=true,
    linkcolor=black,
    filecolor=magenta,      
    urlcolor=black,
    pdftitle={Overleaf Example},
    pdfpagemode=FullScreen,
    }

\title{Resenha - "Coletivos de pensamento em física nuclear em comparação: César Lattes e físicos do Hemisfério Norte nos anos 1940"}
\author{João Pedro Silva dos Santos}
\date{29 de outubro de 2021}

\begin{document}

\maketitle

{\ttfamily{Matéria}}: Filosofia da física

{\ttfamily{Tarefa}}: Escrever uma resenha sobre o artigo: Coletivos de pensamento em física nuclear em comparação: César Lattes e físicos do Hemisfério Norte nos anos 1940

\section*{- Título do artigo:}

Coletivos de pensamento em física nuclear em comparação: César Lattes e físicos do Hemisfério Norte nos anos 1940

\section*{-  Bibliografia:}

TAVARES, Heraclio. Coletivos de pensamento em física nuclear em comparação: César Lattes e físicos do Hemisfério Norte nos anos 1940. [S. l.], 2021. Nota de aula.

\section*{- Objetivo do autor:}

Este artigo propõe analisar comparativamente a prática científica em Física Nuclear de César Lattes na década de 1940 às práticas científicas de físicos vinculados a laboratórios nos EUA e na Europa, que investigavam os mesmos problemas tratados pelo cientista brasileiro.

\section*{- Questões do autor:}

Sendo este artigo um relato histórico não existem questões mas sim uma certa linha do tempo onde o autor descreve alguns dos problemas dos laboratórios, das políticas envolvidas na ciência entre outros.

\section*{- Referencial teórico:}

O presente artigo é baseado no trabalho de pesquisa de mesmo título que teve como pesquisador responsável Ivã Gurgel, professor no Instituto de Física da USP, este trabalho foi um projeto de pós-doutorado cuja bolsa teve vigência de 01 de setembro 2018 até 31 de outubro de 2021. Foi um projeto que “através da análise comparativa das habilidades para o manuseio de conjuntos de instrumentos científicos, formados a partir de estilos de pensamento (FLECK, 2010) distintos, pretendemos entender de que forma a cultura experimental em Física Nuclear desenvolvida no Brasil contribuiu para a detecção e identificação dos mésons pi e mi por Lattes, usando emulsões nucleares como detector, conjugadas a raios cósmicos, em 1947, e a um acelerador de partícula, em 1948. ”

\section*{- Quais os principais assuntos abordados ao longo do texto:}

Este é um artigo que faz um traçado histórico sobre como era a física antes da segunda guerra, durante a guerra e depois da guerra, enquanto faz isso realiza uma análise dos trabalhos de Cesar Lattes pelo mundo. Enquanto faz esse traçado o autor passa pelos problemas enfrentados pelos físicos experimentais da época tanto nas limitações de experimentos quanto nas novas tecnologias que viam surgindo. Além das questões de como interpretar os novos dados, sabendo que cada experimento ou cada forma de se obter dados tem que ser interpretada da forma certa. Sendo assim o autor nos mostra um pouco como foi a evolução dos instrumentos e suas técnicas bem como nós fala sobre a importância de ter alguém que soubesse interpretar os dados que eram fornecidos.

\section*{- Quais as conclusões do autor:}

Deixo em palavras do autor:

"A ideia de ciência que Lattes esposava sobre os inobserváveis se liga à causa de relações visíveis ou, no mínimo, manuseáveis. Sua visão, quando perguntado sobre os quarks, era clara: “Só interessa o que você pode detectar ou o que você pode induzir a partir do que você detectou.” Isso está ligado a um problema que o físico brasileiro via para identificar as participações individuais em projetos de grande escala: “O top [quark] parece que agora foi detectado, o trabalho é bem grosso, mas as primeiras cinco páginas são 333 assinaturas. Isso é física? Espera aí! Com que cada um contribui? Quem é o arquiteto do trabalho?” Novamente temos a questão do artesanal diante do industrial. Como Lattes foi um dos últimos a receber treinamento artesanal na física nuclear, ainda antes da existência de grandes aceleradores, ele entendia o homem como parte da natureza que analisava, não delegando aos instrumentos a independência que eles, na realidade, não possuem. Sua investigação experimental incluía a produção, coleta e interpretação dos dados de maneira individual ou em pequenos grupos, usando instrumentos para auxiliá-lo neste conjunto de tarefas, não para que eles executassem alguma delas em seu lugar. Isso ajuda a explicar a virtuosidade da sua prática com as diferentes ferramentas que manejou e a confusão semântica que experimentou ao se conscientizar da relação que possuía com os inobserváveis, que, provavelmente, tinha naturalizado.
Não sabemos se Lattes chegou a pensar sobre isso, mas, no fundo no fundo, seu trabalho nos permite olhar para as diferentes formas de representação da realidade e suas transformações no tempo, levando-nos a um exercício de autoconhecimento coletivo, onde a variedade de representações possíveis é reflexo da variedade de formas de existência humana. Isso indica que devemos tratar a natureza a partir de uma perspectiva pluralista, que não deve ser confundida com um relativismo ingênuo de inexistência de padrões. O que há, ao contrário, são vários deles, sobre os quais a história e a filosofia das ciências tem o dever de se debruçar."


\section*{- Minhas opiniões:}

Como um texto bem completo não muito a ser dito. É um texto interessate que nos leva uma viagem através da história de um dos principais ciêntistas do país bem como de uma parte da história da física.

\end{document}
