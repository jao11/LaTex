\documentclass [a4paper, 12pt]{article}

\usepackage[T1]{fontenc}
\usepackage[utf8]{inputenc}
\usepackage[letterpaper,top=2cm,bottom=2cm,left=3cm,right=3cm,marginparwidth=1.75cm]{geometry}
\usepackage[brazil]{babel}
\usepackage{amsmath}
\usepackage{indentfirst}
\usepackage{calligra}

\title{Resenha "Filosofia Natural, Física Teorica, Metafísica: Da física dos filósofos antigos à filosofia dos físicos modernos "}
\author{João Pedro Silva dos Santos}
\date{22 de setembro de 2021}

\begin{document}

\maketitle

Matéria: Filosofia da física

Tarefa: Escrever uma resenha sobre o artigo: Filosofia Natural, Física Teorica, Metafísica: Da física dos filósofos antigos à filosofia dos físicos modernos.

\section*{- Título do texto:}

Filosofia Natural, Física Teorica, Metafísica: Da física dos filósofos antigos à filosofia dos físicos modernos.

\section*{-  Bibliografia:}

DA SILVA, Vinícius. Filosofia Natural, Física Teorica, Metafísica: Da física dos filósofos antigos à filosofia dos físicos modernos. Perspectivas - UFT, [s. l.], ano 1, v. 6, p. 274-297, 16 jul. 2021. DOI https://doi.org/10.20873/rpv6n1-97. Disponível em: https://ava.pr1.uerj.br/pluginfile.php/283099/mod_resource/content/1/Filosofia%20Natural%2C%20Fi%CC%81sica%20Teo%CC%81rica%20e%20Metafi%CC%81sica%20-%20Revista%20Perspectivas%281%29.pdf. Acesso em: 22 set. 2021.

\section*{- Objetivo do autor:}

O objetivo deste artigo é explorar as variadas e complexas relações entre ciência e metafísica.

\section*{- Questões do autor:}

\begin{itemize}
    \item O que é a metafísica?
    \item Como compreendemos o metafísica?
    \item Podemos notar a presença da metafísica nas ciências?
    \item Como se deu a problematização da metafísica nas ciências naturais?
    \item A metafísica é inevitável para as ciências naturais?
    \item Em que se baseia a ciência?
    \item Como é feita a física?
    \item “Como garantir que a ciência não é apenas um jogo matemático de palavras e símbolos, um conjunto de modelos ficcionais, mas o modo mais eficiente e rigoroso de conhecermos, ainda que aproximadamente, o real? ”
\end{itemize}

\section*{- Referencial teórico:}

O artigo nasceu das pesquisas do autor no Physikós – Estudos em História e Filosofia da Física e da Cosmologia (FACH-UFMS).

\section*{- Quais os principais assuntos abordados ao longo do texto:}

Neste artigo trataremos de saber como a metafísica pode ser concebida de um conjunto de pressupostos e princípios básicos. Seguimos falando de como ao longo da história as questões metafísicas perpassaram em diversos níveis da investigação natural, e veremos que a metafísica nas ciências naturais foi objeto de debates e disputas. E por fim veremos como alguns dos mais famosos físicos da história recente vão considerar a metafísica como um fundamento que não se pode renunciar.

\section*{- Quais as conclusões do autor:}

Em palavras do próprio autor: 

“ Demonstramos que Ciência e Metafísica são atividades intelectuais entrelaçadas, e, embora tenhamos dado ênfase ao modo como a metafísica é vital para a ciência, pensamos que a relação é reciproca, e que a melhor metafísica de todas as épocas foi aquela feita por pensadores profundamente cientes dos resultados mais expressivos da ciência em seu tempo, pensadores com sólida formação em filosofia natural ou, seja como for, em ciência, a começar por Aristóteles. Isto significa que Ciência e Metafísica, Metafísica e Ciência nutrem-se mutuamente, a tal ponto que é impossível eliminar uma sem secar a outra. “

\section*{- Minhas opiniões:}

Este é um artigo que durante toda a discussão nos mostra, através da análise histórica da formação de algumas teorias, como se deu a construção das ciências naturais enquanto elas dialogavam com o “realismo” e o “irreal” para que pudéssemos chegar a conclusões sobre o mundo que vivemos. Logicamente sempre pensando em como garantir o que é ou não real.

É um artigo interessante pois nos explica como nós, mesmo que sem perceber, transformamos a metafísica em ciência. Ao fazer isso ele nos mostra a necessidade  da metafísica no meio cientifico.

\end{document}
