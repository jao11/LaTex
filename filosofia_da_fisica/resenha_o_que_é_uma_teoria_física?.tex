\documentclass [a4paper, 12pt]{article}

\usepackage[T1]{fontenc}
\usepackage[utf8]{inputenc}
\usepackage[letterpaper,top=2cm,bottom=2cm,left=3cm,right=3cm,marginparwidth=1.75cm]{geometry}
\usepackage[brazil]{babel}
\usepackage{amsmath}
\usepackage{indentfirst}
\usepackage{calligra}

\title{Resenha "O que é uma teoria física?"}
\author{João Pedro Silva dos Santos}
\date{14 de setembro de 2021}

\begin{document}

\maketitle
Matéria: Filosofia da física

Tarefa: Escrever uma resenha sobre o texto O que é uma teoria física? Françoise Basibar.
\section*{- Título do texto:}

O que é uma teoria física?

\section*{-  Bibliografia:}

BASIBAR, Françoise. O que é uma teoria física?. In: GIL, Fernando. A ciência tal qual se faz. [S. l.: s. n.], 1999. p. 247- 268

\section*{- Objetivo do autor:}

O autor se propõe a mostrar que a expressão "teoria física" não faz sentido fora do contexto da sua história.

\section*{- Questões do autor:}

\begin{itemize}
    \item O que havia antes da teoria física?
    \item Qual o porquê do termo teoria física?
    \item Como unir as diversas teorias físicas umas às outras?
    \item Como Duhem estrutura sua teoria física?
    \item Para avançar a teoria física deve avançar por postulados, ou por analogias?

\end{itemize}
\section*{- Referencial teórico:}

As teorias em que o autor se baseia no início do texto são basicamente as maneiras como os principais autores da física embasam suas teorias, e depois usa a teoria de Duhem para tentar criar um padrão de como as teorias físicas se desenvolvem.

\section*{- Quais os principais assuntos abordados ao longo do texto:}

O texto é iniciado com uma análise de quando o formalismo matemático se iniciou, ou seja, com Galileu, o autor analisa e comenta sobre cientistas como Kepler, Newton e Ampère e sobre a maneira como eles montaram e nomearam as suas respectivas teorias. E segue para falar de Fourier que segundo o autor tem sua teoria como a virada fundamental para o nascimento do termo “teoria física” ao criar uma teoria para o calor.

Seguimos o texto falando das teorias e de como a partir de Fourier vemos o nascimento da teoria física, uma vez que nem todos os fenômenos respeitam a mecânica, começa-se a falar sobre Maxwell e do aparecimento de dois formalismos. Então aparece Duhem que define uma teoria física como o resultado de quatro operações, e vemos que os pontos de vista de Maxwell e Duhem divergem.

Para o próximo tópico temos Einstein que se se utiliza da maneira de Maxwell e da maneira de Duhem para formular suas teorias, ou seja, dois métodos para duas teorias que passaram a ser aceitas. Mas a história continua com o surgimento da teoria quântica e também o surgimento de uma adaptação de como se faz uma teoria física.


\section*{- Quais as conclusões do autor:}

Para a conclusão do autor vemos que a cada tópico ele conclui algo que no próximo tópico sofrerá uma modificação ou atualização, isso levando em consideração o avanço da ciência na história, como já é dito logo nas primeiras frases do texto o termo teoria física não faz sentido sem o seu contexto histórico. Pensando nisso podemos dizer que ele conclui que a teoria física (quântica) é uma linguagem metafórica.

\section*{- Minhas opiniões:}

Temos em mãos um texto que nos traz uma linha do tempo de como a teoria física se desenvolveu, podemos ver em tudo que se diz que por mais que tentemos definir um método para se criar uma teoria física, a maneira como elas são criadas vem sendo modificada, ou atualizada, para poder responder os novos questionamentos que surgem a partir das conclusões feitas nas antigas teorias. Sendo assim podemos dizer que por mais que tentemos criar esse método hoje ele provavelmente será modificado no futuro.

\end{document}
