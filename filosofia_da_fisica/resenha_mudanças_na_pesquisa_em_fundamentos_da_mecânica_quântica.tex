\documentclass [a4paper, 12pt]{article}

\usepackage[T1]{fontenc}
\usepackage[utf8]{inputenc}
\usepackage[letterpaper,top=2cm,bottom=2cm,left=3cm,right=3cm,marginparwidth=1.75cm]{geometry}
\usepackage[brazil]{babel}
\usepackage{amsmath}
\usepackage{indentfirst}
\usepackage{calligra}

\title{Resenha "Mudanças na pesquisa em fundamentos da mecânica quântica, 1950-1990"}
\author{João Pedro Silva dos Santos}
\date{08 de setembro de 2021}

\begin{document}

\maketitle

Matéria: Filosofia da física

Tarefa: Escrever uma resenha o artigo mudanças na pesquisa em fundamentos da mecânica quântica, 1950-1990, Prof. Olival Freire Junior (UFBA))

\section*{-  Título do artigo:}

Mudanças na pesquisa em fundamentos da mecânica quântica, 1950-1990

\section*{-  Bibliografia:}

FREIRE JUNIOR, Olival. Mudanças na pesquisa em fundamentos da mecânica quântica, 1950-1990. Caderno Brasileiro de Ensino de Física, [S. l.], v. 32, n. 2, p. 369-377, 1 ago. 2015.


\section*{- Objetivo do autor:}

Examinar o conjunto de fatores, internos e externos, que condicionaram a mudanças dramáticas na segunda metade do século XX, tendo evoluído de uma posição marginal na agenda de pesquisa da física para uma posição muito bem valorizada.

\section*{- Questões do autor:}

O autor nos expõem a um texto onde ele conta uma história sendo assim as questões do texto aparecem como sendo a duas das questões que foram tratadas pela física quântica ao longo do tempo. Sendo elas:

\begin{itemize}
    \item A possibilidade de completar a teoria quântica com variáveis adicionais, visando recuperar ou o realismo ou o determinismo;
    \item O problema da medição quântica.
\end{itemize}

\section*{- Referencial teórico:}

Este artigo é uma tradução de “From the margins to the mainstream: Foundations of quantum mechanics, 1950-1990,” publicado em Annalen der Physik, 527(5-6), A47-A51, 2015. A seção 5 foi ampliada no artigo ora publicado. A tradução foi preparada com o auxílio de Giselle Ferraz.
\section*{- Quais os principais assuntos abordados ao longo do texto:}

Nosso texto se inicia ao falar das questões que perturbavam a física na década de 50 e de como o seu desenvolvimento, através de muitos custos, levou a física quântica a se tornar um ramo mais respeitado da ciência. Ele segue para o tópico dois citando e comentando um pouco sobre que questões foram essas.

Já no terceiro tópico o autor nos conta sobre alguns físicos que arriscaram suas carreiras e reputações para que as pesquisas sobre os fundamentos da mecânica quântica pudessem prosseguir. E por fim na quarta parte do artigo comenta também um pouco sobre as políticas da época e como elas ajudaram a tornar essas pesquisas possíveis.

\section*{- Quais as conclusões do autor:}

Nas palavras do próprio autor:

“A astúcia da história é que na tentativa de atacar os fundamentos da mecânica quântica, isto é, exibir limites na validade da teoria quântica, eles acabaram consolidando e desenvolvendo esta teoria física e a teoria quântica entrou no século 21 mais corroborada que nunca. ”

“A esta altura vamos apenas enunciar algumas destas conclusões, sem desenvolvê-las. Trata-se de uma história que nos permite valorizar uma leitura sociológica da mudança na ciência. O uso da noção de campo científico, na perspectiva de Pierre Bourdieu, ajuda-nos a compreender a estratégia dos dissidentes como uma estratégia de subversão (contraposição à estratégia de conservação) globalmente bem sucedida. Esta história nos mostra o valor da pluralidade na prática da ciência. A diversidade de interpretações da teoria quântica, sem resultados empíricos que permitam uma distinção entre estas interpretações, tem sido frutífera para o desenvolvimento da pesquisa em física. Trata-se também de um exemplo de imbricamento de contexto e conteúdo na mudança da ciência, evidenciando a limitação da dicotomia entre internalismo e externalismo na historiografia da ciência. Por fim, é uma história que nos revela uma lição filosófica, a da existência de conflitos contínuos, e produtivos, entre instrumentalismo e realismo na mudança da ciência. ”

\section*{- Minhas opiniões:}

Este é um artigo muito intrigante que nos mostra como é difícil para um novo ramo da ciência se firmar como sendo boa ciência, tendo que enfrentar nomes antigos e já consolidados, nomes que não querem ver as coisas por ângulos diferentes e por isso muitas vezes consideram as novas teorias como não sendo nem física. Graças a coragem de jovens cientistas que os fundamentos da quântica foram e estão sendo desenvolvidos.

É interessante ver como as politicas influenciaram nesse desenvolvimento, como o trecho do artigo a seguir nos mostra: "Como parte das batalhas ideológicas da época, filósofos e físicos soviéticos criticaram a interpretação usual da mecânica quântica, ou seja, a complementaridade, por ser subjetivista. Ao contrário de outras críticas culturais, esta ressoou com críticas semelhantes feitas no Ocidente". Pois pudemos ver mais uma vez como uma guerra ajudou a criar novas ideias. 


\end{document}
