\documentclass{article}
\usepackage[utf8]{inputenc}
\usepackage[letterpaper,top=2cm,bottom=2cm,left=3cm,right=3cm,marginparwidth=1.75cm]{geometry}

\title{Resenha "John Ziman e a ciência pós-acadêmica:
consensibilidade, consensualidade
e confiabilidade"}
\author{João Pedro Silva dos Santos}
\date{26 de Augusto de 2021}

\begin{document}

\maketitle

Matéria: Filosofia da física


Tarefa:Escrever uma resenha do artigo "John Ziman e a ciência pós-acadêmica: consensibilidade, consensualidade e confiabilidade."


\section*{- Título do artigo:}

John Ziman e a ciência pós-acadêmica: consensibilidade, consensualidade e confiabilidade

\section*{- Bibliografia:}

dos Reis, Verusca e Videira, Antonio. John Ziman e a ciência pós-acadêmica: consensibilidade, consensualidade e confiabilidade. Scientiæ Zudia, São Paulo, v. 11, n. 3, p. 583-611, 2013

\section*{- Objetivo dos autores:}

"Este artigo tem como objetivo central mostrar, a partir de uma discussão das principais teses de Ziman, que as alterações ocorridas no modo de produção da ciência contemporânea têm implicações filosóficas, históricas e sociais que alteraram as relações entre ciência, tecnologia e sociedade" (Ziman, 1996b; 2000).

\section*{- Questões dos autores:}

Os autores não formulam questionamentos em vez disso expõe algumas situações que John Ziman propos em suas teorias.

Algumas questões de Ziman:

“De que maneira os cientistas transmitem seus ensinamentos, se comunicam, promovem, criticam, honram, dão ouvidos e patrocinam uns aos outros? Qual é a natureza da comunidade da qual eles fazem parte?”

\section*{- Referencial teórico:}

As referências dos autores são os trablhos de John Ziman que falam sobre filosofia. 

\section*{- Quais os principais assuntos abordados ao longo do texto:}

Os assuntos serão abordados em seções:

- Seção 1:  Será mostrado como os questionamentos advindos da sua prática como físico levaram Ziman a defender a tese de que só é possível compreender a natureza da ciência caso a vejamos como uma atividade eminentemente social e cooperativa.

- Seção 2: Haverá a analise dos conceitos de “consensibilidade” e “consensualidade” por ele cunhados e a relação destes com o grau de confiabilidade que a sociedade pode ter na ciência. 

- Seção 3: Foi abordado o surgimento da ciência pós-acadêmica, que se deu através do processo de coletivização da ciência, bem como as suas principais características e consequências para a relação entre ciência e sociedade.

\section*{- Quais as conclusões dos autores:}

Os autores terminam mostrando suas conclusões sobre o trabalho e a vida de Ziman como mostrados nos trechos a baixo:

"Por ter uma visão ampla sobre a atividade científica, Ziman estava interessado na elaboração de um programa de ensino de ciências que, ao seguir as teses centrais das novas correntes de estudos sociais da ciência, reformulasse concepções positivistas e fundacionistas sobre o conhecimento científico."

"Ele percebeu que as transformações trazidas por uma ciência pós-acadêmica teriam consequências tanto para a prática da ciência (novas formas de organização e gestão) quanto para os princípios sociológicos (éthos da ciência) e epistemológicos (busca da verdade e da objetividade como um ideal regulador)."

"...uma vez que as diversas concepções descritivas implicam normas relativas à interface ciência e sociedade, o físico inglês sempre defendeu a necessidade de cientistas, políticos e leigos preocuparem-se – em conjunto –com o surgimento de uma ciência pós-acadêmica, que diminuiria nossa capacidade cognitiva de criar novos mapas do mundo. Ziman, em suma, jamais deixou de acreditar na força e na relevância do conhecimento “puro” e acadêmico, o que o torna um defensor da necessidade de a ciência ter um éthos próprio, mantendo algumas tradições caras à cultura acadêmica."

\section*{- Minhas opiniões:}

Um texto que expõe algumas das experiências e observações de John Ziman, fala sobre suas teorias para a formação da ciência e por que devemos crer na ciência. É um texto que nos instiga a saber um pouco mais sobre Ziman e suas teorias.  

\end{document}
