\documentclass [a4paper, 12pt]{article}

\usepackage[T1]{fontenc}
\usepackage[utf8]{inputenc}
\usepackage[letterpaper,top=2cm,bottom=2cm,left=3cm,right=3cm,marginparwidth=1.75cm]{geometry}
\usepackage{hyphenat}
\hyphenation{mate-mática recu-perar}
\usepackage[brazil]{babel}
\usepackage{amsmath}
\usepackage{indentfirst}
\usepackage{calligra}
\usepackage{hyperref}

\hypersetup{
    colorlinks=true,
    linkcolor=black,
    filecolor=magenta,      
    urlcolor=black,
    pdftitle={Overleaf Example},
    pdfpagemode=FullScreen,
    }

\title{Resenha - "Tradução do artigo Eigenart und Wandlungen physikalischer Erkenntnis (1965) de Paul Feyerabend"}
\author{João Pedro Silva dos Santos}
\date{06 de outubro de 2021}

\begin{document}

\maketitle

{\ttfamily{Matéria}}: Filosofia da física

{\ttfamily{Tarefa}}: Escrever uma resenha sobre o artigo: Tradução do artigo Eigenart und Wandlungen physikalischer Erkenntnis (1965) de Paul Feyerabend

\section*{- Título do artigo:}

Tradução do artigo Eigenart und Wandlungen physikalischer Erkenntnis (1965) de Paul Feyerabend

\section*{-  Bibliografia:}

LUZ, Rafael. Tradução do artigo Eigenart und Wandlungen physikalischer Erkenntnis (1965) de Paul Feyerabend. -, [s. l.], 2021.

\section*{- Objetivo do autor:}

Traduzir o artigo  Eigenart und Wandlungen physikalischer Erkenntnis (Peculiaridade e mudança no conhecimento físico), que tem como objetivo fazer uma crítica ao processo de elaboração de teorias físicas, em especial no século XX.

\section*{- Questões do autor:}

\begin{itemize}
    \item O que há de novo no desenvolvimento das ideias e no conhecimento físico?
    \item Em que consiste o progresso das ciências?
    \item Em que consiste a diferença entre a visão de mundo mítica das culturas pré-gregas, assim como os mitos atuais, por um lado, e as novas ideias jônicas, por outro?
\end{itemize}

\section*{- Referencial teórico:}

"Apesar de publicado em 1965, este artigo é a reprodução de uma palestra conferida por Feyerabend em 1963, na estação de rádio e televisão RIAS (Rundfunk im amerikanischen Sektor), uma das mais populares rádios da época, transmitida na Alemanha Ocidental durante a ocupação do país, após a queda do regime nazista. A rádio foi criada por autoridades estadunidenses em 1946, era financiada pela Agência Central de Inteligência (CIA, a sigla em inglês) e supervisionada pela Divisão de Controle de Informação (ICD, a sigla em inglês), uma divisão do Amt der Militärregierung für Deutschland (Gabinete do Governo Militar para a Alemanha) dos EUA."

\section*{- Quais os principais assuntos abordados ao longo do texto:}

"Dividido em três seções denominadas As três diferenças, Invenção do método científico e A situação atual, mais a seção introdutória, Feyerabend faz uma breve comparação entre a forma pela qual filósofos naturais pré-socráticos (em especial os jônicos) e físicos como Albert Einstein construíam seus conhecimentos sobre a natureza, em relação à física do século." 

\section*{- Quais as conclusões do autor:}

 "Através da crítica modificá-las e não excluir nenhum
elemento físico dessa crítica? Ou devemos participar da construção gradual de um sistema 
de pensamento que permite segurança, e que nada mais é do que um mito moderno? Esta
é a decisão fundamental que um físico deve tomar atualmente. A existência [dessa
necessidade de decisão] mostra que a física, longe de erigir um sistema de conhecimento
objetivo, elando-se acima das disputas cotidianas, está imersa nelas. É ideologicamente
limitada. E embora o surgimento de um novo mito, rico em fórmulas, seja algo
inquietante, nós podemos, no entanto, nos consolar em saber que cabe a nós nos livrarmos
dele porque a física, como qualquer área do conhecimento em geral, não nos é imposta
externamente, mas é inteiramente nossa própria obra."

\section*{- Minhas opiniões:}

Este é um artigo que nos mostra um pouco do conhecimento da construção cientifica, acredito que esse artigo nos mostra como os metodos cientificos tentam fortalecer as teorias  físicas.

\end{document}
