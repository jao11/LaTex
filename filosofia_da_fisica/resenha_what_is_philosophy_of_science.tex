\documentclass{article}
\usepackage[utf8]{inputenc}
\usepackage[letterpaper,top=2cm,bottom=2cm,left=3cm,right=3cm,marginparwidth=1.75cm]{geometry}

\title{Resenha "what is philosophy of science?"}
\author{João Pedro Silva dos Santos}
\date{21 de Augosto de 2021}

\begin{document}

\maketitle

Matéria: Filosofia da física


Tarefa: Escrever uma resenha sobre o vídeo, apresentado parcialmente durante a aula de 12/08/2021 "Closer to Truth: What is Phylosophy of Science" 


\section*{- Título do vídeo:}

What is Philosophy of Science?

\section*{- Bibliografia:}

WHAT is Philosophy of Science? | Episode 1611 | Closer To Truth. [S. l.], 16 abr. 2020. Disponível em: https://www.youtube.com/watch?v=IvwkMxgahA4. Acesso em: 21 ago. 2021.

\section*{- Objetivo dos autores:}

Através de uma conversa entre ciência e filosofia os autores do documentário buscam uma resposta para o que é filosofia da ciência, e para saber se, hoje, a ciência deve valorizar a filosofia?

\section*{- Questões dos autores:}

A ciência pode descobrir todas as verdades? Ou existem verdades além da ciência? O que é ciência? Um processo? Uma maneira de pensar? Como chegar à essência da ciência, discernir como ela realmente funciona? 

\section*{- Referencial teórico:}

O vídeo apresenta entrevistas com Simon Blackburn, John Hawthorne, Rebecca Newberger Goldstein, John Searle e Daniel Dennett. 

\section*{- Quais os principais assuntos abordados ao longo do vídeo:}

Os assuntos do vídeo são abordados como uma série de entrevistas cujos temas da entrevistas são esses: 

  - A ciência da a confiança para entender como o mundo funciona mas como posso ter confiança na ciência.
  
  - Qual o por que da filosofia da ciência.
  
  - Historicamente, a filosofia não preparou o caminho para as ciências.
  
  - Quais são algumas das contribuições que a filosofia pode dar a ciência moderna.
  
  - A filosofia beneficia a ciência ao esclarecer maneiras de pensar, como as questões são descritas, como as evidências são consideradas, como as teorias são estruturadas. Mas como essas coisas acontecem? Como a filosofia opera sua suposta "mágica"?
  
  - Qual é o valor profundo da filosofia. Qual é o valor da filosofia no mundo moderno.
  
  - Por que muitos cientistas a rejeitam ou ignoram a filosofia.
  
  
\section*{- Quais as conclusões dos autores:}

Respondendo a pergunta, hoje, a ciência deve valorizar a filosofia? O autor afirma que sim, mas apenas se beneficiar a ciência. E apresenta cinco possiveis benefícios:
  
  1. A filosofia esclarece os tipos de perguntas que a ciência faz, permitindo respostas mais precisas.
  
  2. A filosofia expande os tipos de perguntas que a ciência pode fazer, ampliando o escopo dos problemas que a ciência pode resolver.
  
  3. A filosofia melhora o desenho experimental e a robustez da evidência, fortalecendo o método científico.
  
  4. A filosofia define os limites entre a ciência e a pseudociência, marcando certos campos ou reivindicações como não científicos.
  
  5. A filosofia estabelece os limites da ciência, distinguindo a ciência da não ciência.
  
E termina dizendo que uma análise filosófica cuidadosa é vital, e complementa com a frase. Embora a ciência seja o rei, a filosofia pode firmar a coroa mantendo a ciência mais perto da verdade.
  
\section*{- Minhas opiniões:}

Segundo as minhas opiniões a filosofia da ciência é algo que vai existir sempre que os cientistas precisarem  traduzir suas descobertas para a linguagem comum, quando precisarem entender os dados e formular interpretações para eles, sendo assim por mais que alguns cientistas digam que a filosofia é inutil eles acabam fazendo filosofia. O que pode ocorrer é que sem o auxílio de um filósofo treinado eles podem ir por um caminho que historicamente já foi desacreditado e perder algum tempo até perceberem os erros.

\end{document}
