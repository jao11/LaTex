\documentclass [a4paper, 12pt]{article}

\usepackage[T1]{fontenc}
\usepackage[utf8]{inputenc}
\usepackage[letterpaper,top=2cm,bottom=2cm,left=3cm,right=3cm,marginparwidth=1.75cm]{geometry}
\usepackage[brazil]{babel}
\usepackage{amsmath}
\usepackage{indentfirst}
\usepackage{calligra}

\title{Resenha "O que é a filosofia da física?"}
\author{João Pedro Silva dos Santos}
\date{05 de setembro de 2021}

\begin{document}

\maketitle

Matéria: Filosofia da física

Tarefa: Escrever uma resenha da aula "O que é Filosofia da Física?"  Osvaldo Pessoa Jr.(USP)

\section*{-  Titulo do artigo:}

O que é Filosofia da Física?

\section*{-  Bibliografia:}

OSVALDO Pessoa Jr. Aula para o curso de Filosofia da Física, UERJ, [s. l.], 26 ago. 2021. Disponível em: https://ava.pr1.uerj.br/pluginfile.php/265346/mod_resource/content/1/O-que-e-FiFi-UERJ-1.pdf. Acesso em: 4 set. 2021

\section*{- Objetivo do autor:}

Montar uma aula que possa elucidar a filosofia da física para os alunos do curso de física da UERJ.

\section*{- Questões do autor:}

\begin{itemize}
    \item O que é filosofia da física?
    \item Em que se concentra os estudos da física e da filosofia da ciência?
    \item A filosofia tem alguma utilidade para o físico?
    \item Como diferenciar o realismo do antirrealismo?
    \item A intuição que temos a respeito do mundo é confiável?
    \item O que é o comportamento complexo?
    \item A questão do sugimento da conciência.
    \item A mente é uma entidade física, ou pode haver conhecimentos físico completo sobre os estados subjetivos?
    \item O que é a física? Como se diferencia de outras áreas?
\end{itemize}

\section*{- Referencial teórico:}

 O material dessa aula foi extraído das Notas de Aula do curso de Filosofia da Física da USP: PESSOA Jr., O. (2020), Filosofia da física clássica, online: https://opessoa.fflch.usp.br/FiFi-20, seções I.1, II.1,3,4,5,6, III.1.


\section*{- Quais os principais assuntos abordados ao longo do texto:}

O autor do texto começa nos mostrando com o que a filosofia da física trabalha, e segue para o tópico dois nos explicando em que áreas os físicos colocam seus esforços de pesquisa, enquanto coloca uma ênfase no trabalho dos filósofos da ciência, sendo mais especifico comenta que a dois tipos de investigação. E termina o segundo tópico com uma discussão sobre se a a filosofia tem alguma utilidade para o físico, sendo que o trabalho do físico geralmente se dá sem a necessidade de reflexões filosóficas sobre sua atividade.

Ele começa o terceiro tópico comentando sobre uma distinção epistemológica fundamental, que aparece frequentemente em controvérsias científicas, é aquela entre “realismo” e diferentes formas de “antirrealismo”, além é claro de elucidar os significados dos dois para os alunos que assistem a aula. Sucintamente, o realismo defende que a ciência pode fazer afirmações sobre entidades ou leis inobserváveis, ao passo que o que chamaremos de fenomenismo (uma forma de antirrealismo) defende que a ciência só deve se ater ao que é observável ou mensurável. No próximo ponto ele começa a falar sobre qual o papel da intuição na física, primeiro se perguntando se a intuição que temos a respeito do mundo é confiável? Enquanto define “intuição” como uma crença que surge de maneira imediata na consciência, ou seja, sem a mediação de uma cadeia de raciocínios.

Para a quinta parte do texto temos o estudo da lista de fenômenos com os quais nunca tivemos contato epistêmico durante a evolução, esboçada na seção anterior, deixamos de lado o domínio do muitíssimo complicado. Onde teremos o nome complexidade associado aos sistemas de muitas partículas e aos sistemas de poucas partículas. Além de falar do domínio mesoscópico. Que passa a ter um problema conceitual especialmente agudo que surge neste domínio é o do surgimento da consciência, problema esse que é discutido no tema seis. 

Já no tópico sete o autor volta a física em si e à separa em três maneiras de caracteriza-la, maneiras essas que são os objetos de estudo, os métodos utilizados e a sua visão de mundo. E por fim ao juntar todas as informações dos tópicos anteriores introduz algumas ideias de Moritz Schlick. 


\section*{- Quais as conclusões do autor:}

Para a conclusão o autor ele diz o seguinte, “Na seção 6, ao nos perguntarmos se a mente é uma entidade física, ou se pode haver conhecimento físico completo sobre os estados subjetivos, discutimos implicitamente os limites do objeto de estudo da Física (ponto 1, acima). Tradicionalmente, separa-se o físico (objetivo) e o mental (subjetivo), de tal maneira que o “físico” se referiria apenas ao mundo dos objetos externos à nossa mente. Mas isso impossibilitaria a descrição física da mente, que pode ser almejada pela visão de mundo física (ponto 3). O fisicismo gostaria de incorporar as “qualidades” dentro da visão de mundo física, mas isso parece envolver métodos de conhecimento (ponto 2) que estão fora da abordagem linguístico-quantitativa da física. “ 

No fim termina comentando sobre a visão de Moritz Schlick, e fala de algumas conclusões dele. “Na visão fisicista de Schlick, “conceitos espaço-temporais podem ser usados para descrever qualquer realidade arbitrária, sem exceção, incluindo a realidade da consciência” (p. 295). A formulação de Schlick permite fazer uma distinção entre dois tipos de propriedades físicas. O primeiro tipo são as propriedades descritas pelas teorias físicas, como posição relativa, velocidade, massa, momento magnético, número de Reynolds, que são passíveis de quantificação e de simulação computacional, e que correspondem a aspectos relacionais da realidade (proporções entre grandezas reais) mas não a propriedades intrínsecas. O segundo tipo são as propriedades reais dos objetos físicos, que seriam propriedades intrínsecas ou essenciais (“inescrutáveis” ou “qüididades”), e que Schlick chamou de “qualidades”. A tese de que a física só tem acesso às propriedades relacionais das coisas é chamada de estruturalismo na física, sendo que por “estrutura” entende-se o conjunto de relações (causais e de outras espécies) envolvendo um objeto. Esta tese também é chamada de “realismo estrutural”. “

\section*{- Minhas opiniões:}

É uma aula muito interessante que nos instrui muito bem sobre o que é a física e como ela funciona, tenta dar uma melhor visão sobre a parte filosófica da ciência, aquela que trabalha com os problemas não resolvidos da descrição física do mundo.

A descrição da física em quatro pontos principais de estudo tornou a maneira de entender a física muito mais simples do que as antigas divisões que eu sempre havia visto, afinal diferente do que nos foi ensinado a física é uma só, e a separação por matérias cria uma ilusão de que temos diferentes físicas. Além disso mostra que sem as descrições dos eventos fica muito difícil ter uma visão clara dos eventos ainda mais com as dificuldades dos domínios mais complexos do mundo.


\end{document}
