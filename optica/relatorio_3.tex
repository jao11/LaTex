\documentclass [a4paper, 12pt]{article}

\usepackage[T1]{fontenc}
\usepackage[utf8]{inputenc}
\usepackage[letterpaper,top=2cm,bottom=2cm,left=3cm,right=3cm,marginparwidth=1.75cm]{geometry}
\usepackage{hyphenat}
\hyphenation{mate-mática recu-perar}
\usepackage[brazil]{babel}
\usepackage{amsmath}
\usepackage{indentfirst}
\usepackage{calligra}
\usepackage{graphicx}
\usepackage{hyperref}

\hypersetup{
    colorlinks=true,
    linkcolor=black,
    filecolor=magenta,      
    urlcolor=black,
    pdftitle={Overleaf Example},
    pdfpagemode=FullScreen,
    }


\title{\sc{\textbf{Laboratório de Óptica: Tarefa 3}}}
\author{\textit{João Pedro Silva dos Santos}}
\date{\empty}

\begin{document}

\maketitle

{\ttfamily{Matéria}}: Óptica

{\ttfamily{Tarefa}}: Explique, com clareza, a diferença entre interferência e difração.  Para isso, utilize recursos adicionais, como figuras e  desenhos, por exemplo, para que fique clara a diferença entre esses dois fenômenos.

\section{Introdução}

A presente questão à ser respondida é: Qual a diferença entre entre interfência e difração? Para responder esse pergunta deve-se primeiro conhecer bem o que são esses dois fenômenos sendo assim:

\subsection{Definição de Interferência}

Considere que num determinado ponto P do espaço passem duas, ou mais, ondas mecânicas progressivas (ou eletromagnéticas), ao mesmo tempo. Diz-se que ocorre interferência nesse ponto. O efeito resultante nesse ponto, em geral, pode ser obtido através da soma vetorial dos efeitos que cada onda teria sozinha (Princípio da superposição).
As interferências podem ser do tipo construtivas ou destrutivas. Quando o efeito resultante da superposição das ondas no ponto considerado provoca um aumento da amplitude da onda resultante, diz-se que houve interferência construtiva. Quando o efeito resultante provocar uma diminuição da amplitude onda resultante, diz-se, portanto, que houve interferência destrutiva nesse ponto. É importante salientar que para ocorrer interferência destrutiva as ondas devem possuir fases opostas. As figuras a seguir ajudam a entender melhor o caso.

\begin{figure}[!h]
\centering
{\includegraphics[width=1\textwidth]{figura1.png}}
\caption{\label{fig:figura 1} \empty}
\end{figure}

Na figura 1 observa-se um caso de interferência construtiva. Cada pulso segue sua trajetória normalmente após se cruzarem. A representa as amplitudes de cada pulso, enquanto 2A representa a amplitude da onda resultante no ponto onde ocorreu a superposição dos pulsos.

\begin{figure}[!h]
\centering
{\includegraphics[width=1\textwidth]{figura2.png}}
\caption{\label{fig:figura 2} \empty}
\end{figure}

Já na figura 2, dois pulsos com fases invertidas, uma em relação à outra, se superpõem no ponto P e, em seguida, seguem suas trajetórias normalmente. Como o efeito resultante foi uma redução da amplitude da onda resultante, em vez de aumento, diz-se que neste caso ocorreu interferência destrutiva no ponto onde ocorreu a superposição dos pulsos. Caso as amplitudes sejam iguais, no caso de interferência destrutiva, a intensidade da onda (proporcional ao quadrado da sua amplitude) será nula nesse ponto.

\subsection{Definição de Difração}

A difração ocorre quando uma onda consegue contornar algum obstáculo ou passar por algum orifício. É importante ressaltar que a ordem de grandeza do comprimento de onda da onda deve ser aproximadamente igual à ordem de grandeza das dimensões do obstáculo ou da fenda. Se o comprimento de onda da onda for muito maior (ou muito menor) que o tamanho do obstáculo, não ocorrerá difração. Portanto, só é conveniente se referir à difração quando ocorrer alguma dessas situações: uma onda passando por uma fenda ou contornando algum obstáculo.

\begin{figure}[!h]
\centering
{\includegraphics[width=1\textwidth]{figura4.png}}
\caption{\label{fig:figura 4} \empty}
\end{figure}

\newpage

Na figura 3 é possível ver que as ondas planas na superfície de um líquido ultrapassam a fenda e dão origem a ondas circulares. Caso a abertura da fenda seja reduzida, o fenômeno da difração se torna mais acentuado, podendo-se considerar que existiria na abertura uma nova fonte de frentes de onda.


\section{Qual a diferença?}

A interferência fala sobre o encontro de duas ou mais ondas num ponto do espaço no mesmo instante, independentemente de haver obstáculos ou não, i. é, o simples fato de uma onda se superpor à outra já caracteriza a interferência. Por outro lado, a difração está associada ao fato de uma onda contornar ou não um obstáculo (ou atravessar um orifício). É importante observar que após a difração, caso existam dois ou mais obstáculos/orifícios, aparecerá a interferência entre as ondas difratadas, visto que uma onda irá se superpor com outra em algum momento.

Para se ter uma ideia mais clara sobre a diferença pode-se citar palavras de Richard Feynman que diz:

“Ninguém nunca foi capaz de definir a diferença entre interferência e difração satisfatoriamente. É somente uma questão de linguagem, e não há diferenças físicas específicas ou importantes entre elas. Tem-se, entretanto, que difração é o fenômeno devido a um obstáculo, já interferência refere-se mais a uma interação entre dois ou mais fenômenos ondulatórios."

\section{Bibliografia}

PARÂMETROS experimentais e teóricos da radiação X.ESCOLA POLITÉCNICA DA UNIVERSIDADE DE SÃO PAULO, Departamento de Engenharia Metalúrgica e de Materiais.Disponível em:\url{https://edisciplinas.usp.br/pluginfile.php/5289935/course/section/5968297/Aula%208}.  Acesso em: 21 ago. 2021

KNIGHT, Randall. Física 2 [recurso eletrônico]: uma abordagem estratégica / Randall Knight; tradução Iuri Duquia Abreu. – 2. ed. – Dados eletrônicos – Porto Alegre : Bookman, 2009.

Física clássica, 2: termologia, óptica e ondas/Caio Sérgio Calçada, José Luiz Sampaio. — 1. ed. — São Paulo: Atual, 2012.

NUSSENVEIG, Herch Moysés. Curso de Física Básica: Fluidos, Oscilações e Ondas, Calor. 4. Ed. N/c: Edgard Blucher, 2002. 314 p. (Volume 2)

\end{document}
