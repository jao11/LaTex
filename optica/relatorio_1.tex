\documentclass [a4paper, 12pt]{article}

\usepackage[T1]{fontenc}
\usepackage[utf8]{inputenc}
\usepackage[letterpaper,top=2cm,bottom=2cm,left=3cm,right=3cm,marginparwidth=1.75cm]{geometry}
\usepackage[brazil]{babel}
\usepackage{amsmath}
\usepackage{indentfirst}
\usepackage{calligra}
\usepackage{graphicx}

\title{\sc{\textbf{Irradiância de uma fonte quase puntiforme}}}
\author{\textit{João Pedro Silva dos Santos}}
\date{\empty}

\begin{document}

\maketitle

\section*{\hfill{RESUMO}}

Este é um breve relato sobre uma experiência realizada pelo professor Luiz Pinheiro para que a turma de óptica pudesse realizar o tratamento dos dados do mesmo. E assim estudar o uso e a aplicação de fotodetectores e estudar os limites de validade do experimento e os erros inerentes ao mesmo.


\section{Objetivos}

Determinar a dependência luminosa de uma fonte quase puntiforme com a distância ao ponto de detecção.

\begin{itemize}
    \item Estudar o uso e a aplicação de fotodetectores.
    \item Estudar os limites de validade do experimento e os erros inerentes ao mesmo.
\end{itemize}

\section{Introdução}

Se uma fonte de radiação eletromagnética for puntiforme,
esfericamente simétrica e isotrópica, podemos derivar uma expressão
para a irradiância em função da distância até a fonte.

Consideremos que o meio que
circunda a fonte seja não absorvente (sem
perdas). Assim, considerando que a fonte
irradia uniformemente em todas as
direções, essa energia irradiada
atravessará a superfície de uma esfera
concêntrica à fonte.

Fonte puntiforme de luz:

\begin{itemize}
    \item Uma fonte de luz é chamada de puntiforme quando as suas dimensões são desprezíveis em relação à distância do objeto iluminado.
    
    \item Uma fonte de luz é chamada de extensa quando suas dimensões são consideráveis em relação à distância do objeto iluminado.

\end{itemize}

Em algumas situações podemos supor que a fonte seja pontual e
que emite luz isotropicamente, (com igual intensidade em todas as
direções).

A energia das ondas esféricas passam por uma esfera imaginária
de centro em O e raio r. A irradiância I de uma fonte é definida por:

\begin{equation}
  \ I = \dfrac{Pot{\text{ê}}ncia}{{\text{Á}}rea} = \dfrac{P_s}{4\pi r^2} = \dfrac{P_s}{4 \pi} \Bigg(\dfrac{1}{r^{2}} \Bigg) \longrightarrow I \ \alpha \ \dfrac{1}{r^{2}}\  
\end{equation}

Onde: $4\pi r^2$ é a área da esfera centrada na fonte e raio $r$
igual à distância na qual se está mediando a irradiância $I$.


\section{Materiais e métodos}
\subsection{Materiais utilizados}
Os materiais que foram necessários para esse experimento foram: 
 
\begin{itemize}
    \item Banco óptico;
    \item Diafragma;
    \item Fonte de luz;
    \item Fotodetector;
    \item Microamperímetro;
    \item Régua/ Trena;
    \item Anteparo;
    \item Iluminação de baixa intensidade para efetuar a leitura do microamperímetro;
    \item Lápis, caneta, papel, etc. ({\ttfamily{Para anotação dos dados}}).
\end{itemize}

\subsection{Métodos}

Para esse experimento foi utilizado o método de transdução luz/corrente que consiste em detectores construídos com semicondutores, que são baseados no mecanismo de geração de pares, elétrons-lacunas por meio da absorção de fótons irradiados para dentro do material.

\subsection{Procedimento experimental}
Os procedimentos foram divididos em três partes:

\subsubsection{Preparação:}

\begin{enumerate}
    \item  Foi Posicionada a fonte de luz e o fotodetector sobre o trilho;
    \item Depois colocado o diafragma entre a fonte de luz e o detector;
    \item Alinhados os componentes de forma que a luz ilumine completamente a superfície útil do detector;
    \item Foi variada a posição do detector, mantendo a fonte e o diafragma fixos.
\end{enumerate}

\subsubsection{Coleta de dados:}


\begin{enumerate} 
\item Então foram feitas 22 medidas de r (entre o detector e o diafragma) e os correspondentes valores de corrente elétrica.
\end{enumerate}


\subsubsection{Tratamento de dados:}

\begin{enumerate} 
\item Usando o exel para o tratamento de dados e os gráficos;

\item Foram feitos três gráficos: I $\times$ r, I $\times$ 1/r² e $\log(I)$ $\times$ $\log(r)$;

\item Então determinando no gráfico log-log o coeficiente angular da reta;

\item E por fim foi comparado este valor com o valor teórico.

\end{enumerate}

\section{Resultados e discusão}

\subsection{Dados e gráficos}

A partir do experimento foram obtidos os seguintes dados:

\begin{table}[h]
\centering
\caption{Tabela de dados}
\vspace{0.5cm}
\begin{tabular}{|c|c|c|c|c|c|} \hline
Medida & $r(cm)$ & $I(\mu A)$  \\ % Note a separação de col. e a quebra de linhas
\hline                               % para uma linha horizontal
1  & 1,5       & 211,0  \\  \hline 
2  & 2,5       & 89,0  \\  \hline
3  & 3,5       & 49,0  \\  \hline
4  & 4,5       & 31,0  \\ \hline
5  & 5,5       & 20,0  \\ \hline
6  & 6,5       & 14,2  \\ \hline
7  & 7,5       & 11,4  \\ \hline
8  & 8,5       & 8,5   \\ \hline
9  & 9,5       & 6,9   \\ \hline
10 & 10,5      & 5,8   \\ \hline
11 & 11,5      & 4,8   \\ \hline
12 & 12,5      & 4,0   \\ \hline
13 & 13,5      & 3,5   \\ \hline
14 & 14,5      & 3,0   \\ \hline
15 & 15,5      & 2,8   \\ \hline
16 & 17,5      & 2,0   \\ \hline
17 & 19,5      & 1,7   \\ \hline
18 & 21,5      & 1,3   \\ \hline
19 & 23,5      & 1,1   \\ \hline
20 & 25,5      & 1,0   \\ \hline
21 & 28,5      & 0,9   \\ \hline
22 & 36,5      & 0,4   \\ \hline % não é preciso quebrar a última linha
\end{tabular}
\end{table}

\newpage

Com os dados da tabela 1 foi então criardo o gráfico abaixo para o comportamento da corrente ao afastar a fonte luminosa.




\begin{figure}[!h]
\centering
{\includegraphics[width=1\textwidth]{g1.jpg}}
\caption{\label{fig:g1} \empty}
\end{figure}

Passa-se então aos dados, dados esses que estão na próxima tabela, que foram tratados para formar os gráficos de I(microA) $\times$ (1/ r²)(1 / cm²) e de $\log(I)$ $\times$ $\log(r)$, e colocados logo depois da tabela 2.

\newpage

\begin{table}[h!]
\centering
\caption{Tabela de dados tratados}
\vspace{0.5cm}
\begin{tabular}{|c|c|c|c|c|} \hline
medida & $1 / r^2 (1/cm^2)$& $I(microA)$ & log(r)	& log(I)    \\
\hline
1	    &    0,4444444444	 &  211	    &   0,1760912591	&   2,324282455     \\  \hline 
2	    &    0,16	         &  89	    &   0,3979400087	&   1,949390007     \\  \hline
3	    &    0,08163265306	 &  49	    &   0,5440680444	&   1,69019608      \\  \hline
4	    &    0,04938271605	 &  31	    &   0,6532125138	&   1,491361694     \\  \hline
5	    &    0,03305785124	 &  20	    &   0,7403626895	&   1,301029996     \\  \hline
6	    &    0,02366863905	 &  14,2	&   0,8129133566	&   1,152288344     \\  \hline
7	    &    0,01777777778	 &  11,4    &   0,8750612634	&   1,056904851     \\  \hline
8	    &    0,01384083045	 &  8,5	    &   0,9294189257	&   0,9294189257    \\  \hline
9	    &    0,01108033241	 &  6,9	    &   0,9777236053	&   0,8388490907    \\  \hline
10	    &    0,009070294785  &  5,8	    &   1,021189299	    &   0,7634279936    \\  \hline
11	    &    0,007561436673  &  4,8	    &   1,06069784	    &   0,6812412374    \\  \hline
12	    &    0,0064	         &  4	    &   1,096910013	    &   0,6020599913    \\  \hline
13	    &    0,00548696845	 &  3,5	    &   1,130333768	    &   0,5440680444    \\  \hline
14	    &    0,004756242568	 &  3	    &   1,161368002	    &   0,4771212547    \\  \hline
15	    &    0,004162330905	 &  2,8	    &   1,190331698	    &   0,4471580313    \\  \hline
16	    &    0,003265306122	 &  2	    &   1,243038049	    &   0,3010299957    \\  \hline
17	    &    0,002629848784	 &  1,7	    &   1,290034611	    &   0,2304489214    \\  \hline
18	    &    0,002163331531	 &  1,3	    &   1,33243846	    &   0,1139433523    \\  \hline
19	    &    0,001810774106	 &  1,1	    &   1,371067862	    &   0,04139268516   \\  \hline
20	    &    0,00153787005	 &  1	    &   1,40654018	    &   0               \\  \hline
21	    &    0,001231148046	 &  0,9	    &   1,45484486	    &   -0,04575749056  \\  \hline
22	    &    0,0007506098705 &	0,4	    &   1,562292864	    &   -0,3979400087   \\  \hline

\end{tabular}
\end{table}

\newpage


\begin{figure}[!h]
\centering
{\includegraphics[width=1\textwidth]{g2.jpg}}
\caption{\label{fig:g2} \empty}
\end{figure}

\begin{figure}[!h]
\centering
{\includegraphics[width=1\textwidth]{g3.jpg}}
\caption{\label{fig:g3} \empty}
\end{figure}

\subsection{Ajuste Linear}

Pode-se então começar a manipular os dados a partir da seguinte equação:

\begin{equation}
  \ I = {A} \times {r^{(-n)}} = A \times \dfrac{1}{r^{(n)}} \
\end{equation}
\vspace{0.025 cm}

Deve ser adicionado o logaritmo em ambos os lados da equação (2):

\begin{equation}
  \ \log (I) = \log \Bigg( {A \times \dfrac{1}{r^{(n)}} }\Bigg) \times {r^{(-n)}} = \log (A) + \log \Bigg( \dfrac{1}{r^{(n)}} \Bigg)
\end{equation}
\vspace{0.025 cm}
\begin{equation}
  \ \log (I) = \log (A) - n\times\log (r)
\end{equation}

Sabendo que A é uma constante tem-se que $\log(A)$ também é uma constante logo pode-se dizer que $\log(A)$ é igual a b. Assim substituindo b na equação (4) fica-se com:

\begin{equation}
  \ \log (I) = - n\times\log (r) + b
\end{equation}

Que pode ser comparada a uma equação da forma:

\begin{equation}
  \ y = ax + b
\end{equation}

Sabendo disso pode-se então proceder com o processo de ajuste linear.

Utilizando os dados de $\log(I)$ e $\log(r)$, da tabela 2 obtem-se:


\begin{table}[h!]
\centering
\caption{Tabela de dados tratados}
\vspace{0.5cm}
\begin{tabular}{|c|c|c|} \hline

<x>            &   <$\log(r)$>                   &  1,019449053     \\ \hline
<y>            &   <$\log(I)$>                   &  0,7496325205    \\ \hline
<xy>           &   <$\log(I)\times\log(r)$>      &  0,5292414192    \\ \hline
<x²>           &   <$\log(r)^2$>                 &  1,160054545     \\ \hline
<y²>           &   <$\log(I)^2$>                 &  1,020008223     \\ \hline
$\sigma_{xy}$  &    covariância                  &  -0,2349707442   \\ \hline
$\sigma_{x}$   &    desvio padrão $\log(r)$      & 0,3475315421    \\ \hline
$\sigma_{y}$   &    desvio padrão $\log(I)$      & 0,6768007881    \\ \hline
 

\end{tabular}
\end{table}

A partir dos valores da tabela 3 calcula-se então:


\begin{table}[h!]
\centering
\caption{\empty}
\vspace{0.1 cm}
\begin{tabular}{|c|c|c|} \hline

n               &    -1,945473581    \\ \hline
b               &     2,732943721    \\ \hline
$\epsilon_y$    &     0,0319831932   \\ \hline
$\epsilon_x$    &    -0,01643979826  \\ \hline
$\sigma_n$      &    0,01962077934   \\ \hline
$\sigma_b$      &    0,02113272291   \\ \hline

\end{tabular}
\end{table}
\newpage
Desse modo a reta de calibração é expressa como:

\begin{equation}
  \ \log (I) (microA) = -1,945\times\log (r) (cm) + 2,732
\end{equation}

\begin{equation}
  \ \epsilon_y = \epsilon_{_{log(I)}} = 0,031 (microA)
\end{equation}

Tendo como expressão final a junção de (7) com (8). Assim tem-se:

\begin{equation}
  \ \log (I) \pm \big[ \epsilon_{_{log(I)}} \big] = \Big(-1,945\cdots\times\log (r)\Big) \pm (0,031) + (2,732) \pm (0,031)
\end{equation}

Assim foi encontrado um coeficente angular $n$ $\cong$ - 1,945.


\subsection{Erros e compatibilidade}

Como se pode observar em nossa equação (1) tem-se o $n$ valendo -2 que é o valor de referência, como o $n$ encontrado não tem o mesmo valor passasse então ao cálculo de sua compatibilidade.

Usando a relação

\begin{equation}
  \ \Big|n - n_{ref}\Big| < 2\times\sigma_n
\end{equation}

Pode-se dizer se o valor é ou não compativel com a teoria.

Substituindo os valores em (10) fica-se com:

\begin{equation}
  \ \Big|(-1,945\cdots) - (-2)\Big| < 2\times0,019\cdots \rightarrow 0,05452641892 < 0,03924155869
\end{equation}

Como se pode ver o resultado dessa desigualdade é falso. Passa-se a fazer então 

\begin{equation}
  \ \Big|n - n_{ref}\Big| \le 3\times\sigma_n
\end{equation}

Substituindo os valores em (12) fica-se com:

\begin{equation}
  \ \Big|(-1,945\cdots) - (-2)\Big| \le 3\times0,019\cdots \rightarrow  
  \ 0,05452641892 \le 0,05886233803
\end{equation}

Como pode-se observar o resultado dessa desigualdade é verdadeiro. Ou seja ele se encaixa na situação da equação (14) que dará uma resposta inconclusiva à analise.

\begin{equation}
  \  2\times\sigma_n < \Big|n - n_{ref}\Big| \le 3\times\sigma_n
\end{equation}

Possa-se a calcular o erro relativo para que se possa ver qual a proximidade fica a resposta em relação ao valor de referência.

è possivel conseguir esse erro fazendo:

\begin{equation}
  \  E = \dfrac{\Big|n - n_{ref}\Big|}{n_{ref}} \times 100 \%
\end{equation}

Substituindo os valores na equação (15) fica-se com:

\begin{equation}
  \  E = 2,73\%
\end{equation}

Que é o erro relativo.

\section{Conclusões}

Através da elaboração deste relatório se pode ter uma noção de uma aplicação dos fotodetectores para a produção de corrente eletrica ou no caso da irradiância luminosa. Sendo assim pode-se ver que quanto maior a distância menor a irradiância.

Através de manipulações chegou-se numa relação em se pode aplicar o ajuste linear e sendo assim foi realizada a linearização e os testes de compatibilidade para a equação linearizada. Pode-se chegar a conclusão de que os dados nos deram uma equação inconclusiva, porém ao calcular erro relativo viu-se que se que a mesmo continha um erro relativamente pequeno. Portanto pode-se dizer que a experiência foi bem sucedida.

\section{Referências}

PINHEIRO, Luiz. Modelo de relatório. 2021. Nota de Aula (Graduação, Física) - Professor, [S. l.], 2021. pdf.

SANTORO, Alberto; MAHON, José Roberto; DE OLIVEIRA, José Umberto; MUNDIM FILHO, Luiz; OGURI, Vitor; DA SILVA, Wanda. Estimativas e erros em experimentos de física. 2. ed. Rio de Janeiro: Ed UERJ, 2008. 131 p.


\end{document}
